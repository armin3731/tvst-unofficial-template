\documentclass[10pt, a4paper]{article}
\usepackage{tvst}
\usepackage{lipsum} % For filler text
\usepackage{hyperref}
\usepackage{tabularx}

\begin{document}

\journalheader

\papertitle{Analysis of Keratoconus-Related Phenotypes in Two Pcsk1 Mouse Models}

\authorlist{Carol Beatty\textsuperscript{1,*}, Jingwen Cai\textsuperscript{2,3,*}, Hongfang Yu\textsuperscript{2}, Jiong Sun\textsuperscript{1,2}, Yejin Heo\textsuperscript{1}, Keith H. Baratz\textsuperscript{3}, Ashlie A. Bernhisel\textsuperscript{4}, Sanjay V. Patel\textsuperscript{5}}

\affiliations{
\textsuperscript{1} Medical College of Georgia, Augusta University, Augusta, GA, USA \\
\textsuperscript{2} Department of Cellular Biology and Anatomy, Augusta University, Augusta, GA, USA \\
\textsuperscript{3} Department of Ophthalmology, Mayo Clinic, Rochester, MN, USA \\
\textit{(Add remaining affiliations as per the image)}
}

\begin{minipage}[t]{0.35\textwidth}
    \infosection{Correspondence}{}
    \infosection{Received}{}
    \infosection{Accepted}{}
    \infosection{Published}{}
    \infosection{Keywords}{}
    \infosection{Citation}{}
\end{minipage}
\hfill
\begin{minipage}[t]{0.60\textwidth}
    \small\rmfamily\sffamily
    \textbf{Purpose:} A structured abstract of fewer than 250 words is required for articles and should be arranged under the following headings: Purpose, Methods, Results, Conclusions, Translational Relevance (a one-sentence description of how your work bridges the gap between basic research and clinical care). Define abbreviations at first mention, and do not include references. The abstract must be included as part of the main manuscript file.\\
    Abstracts are also required for review articles; however, these do not need to be in a structured format.\\
    In addition, authors whose native language is not English may submit a Foreign Language Abstract along with the manuscript file. If the manuscript is accepted, the Foreign Language Abstract will be published with the final published article.
    \par\vspace{0.8em}
    \textbf{Methods:} \par\vspace{0.8em}
    \textbf{Results:} \par\vspace{0.8em}
    \textbf{Conclusions:} \par\vspace{0.8em}
    \textbf{Translational Relevance:} 
\end{minipage}

\begin{multicols}{2}
\sectionbox{Introduction}
In a brief Introduction (don't use any subheadings), provide the research rationale and objectives without extensively reviewing the literature.

\sectionbox{Methods}
In the Methods section, describe the experimental design, subjects used, and procedures followed. Previously published procedures should be identified by reference only. Provide sufficient detail to enable others to duplicate the research. Use standard chemical or nonproprietary pharmaceutical nomenclature. In parentheses, identify specific sources by brand name, company, city, and state or country.

If human subjects were involved in the investigation, the Methods section must confirm that: (1) informed consent was obtained from the subjects after explanation of the nature and possible consequences of the study; and (2) where applicable, the research was conducted in compliance with a suitable accredited Institutional Review Board (IRB) or its equivalent (such as a human ethics committee).

If experimental animals were used in the investigation, the Methods section must confirm adherence to the ARVO Statement for the Use of Animals in Ophthalmic and Vision Research and, where applicable, approval by the appropriate IRB.

\subsection{This is Sub-Section}
lorem ipsum dolor sit amet, consectetur adipiscing elit. Donec a diam lectus. Sed sit amet ipsum mauris. Maecenas congue ligula ac quam viverra nec consectetur ante hendrerit. Donec et mollis dolor. Praesent et diam eget libero egestas mattis sit amet vitae augue. Nam tincidunt congue enim, ut porta lorem lacinia consectetur. Donec ut libero sed arcu vehicula ultricies a non tortor. Lorem ipsum dolor sit amet, consectetur adipiscing elit. Aenean ut gravida lorem. Ut turpis felis, pulvinar a semper sed, adipiscing id dolor. Pellentesque auctor nisi id magna consequat sagittis. Curabitur dapibus enim sit amet elit pharetra tincidunt feugiat nisl imperdiet. Ut convallis libero in urna ultrices accumsan. Donec sed odio eros. Donec viverra
\subsubsection{This is Sub-Sub-Section}
lorem ipsum dolor sit amet, consectetur adipiscing elit. Donec a diam lectus. Sed sit amet ipsum mauris. Maecenas congue ligula ac quam viverra nec consectetur ante hendrerit. Donec et mollis dolor. Praesent et diam eget libero egestas mattis sit amet vitae augue. Nam tincidunt congue enim, ut porta lorem lacinia consectetur. Donec ut libero sed arcu vehicula ultricies a non tortor. Lorem ipsum dolor sit amet, consectetur adipiscing elit. Aenean ut    
\begin{figure*}[ht] % [t] is recommended for double-column floats
    \centering
    \includegraphics[width=1.0\textwidth]{images/placeholder.png}
    \caption{This a sample figure.}
\end{figure*}
% Full-width Table
\begin{table*}[t]
    \caption{This is a sample table.}
    \sffamily
    % \small % Only use if you want to reduce the font size for the table content
    \Large
    \begin{tabularx}{\textwidth}{XXXX}
        Column 1 & Column 2 & Column 3 & Column 4 \\
        \hline
        Data A & Data B & Data C & Data D \\
        Data A & Data B & Data C & Data D \\
        Data A & Data B & Data C & Data D \\                
        Data A & Data B & Data C & Data D \\        
        \hline
    \end{tabularx}
\end{table*}



\end{multicols}

\end{document}