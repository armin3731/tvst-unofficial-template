\documentclass[10pt, a4paper]{article}
\usepackage{tvst}
\usepackage{lipsum} % For filler text
\usepackage{hyperref}

\begin{document}

\journalheader

\papertitle{Analysis of Keratoconus-Related Phenotypes in Two Pcsk1 Mouse Models}

\authorlist{Carol Beatty$^{1,*}$, Jingwen Cai$^{2,3,*}$, Hongfang Yu$^{2}$, Jiong Sun$^{1,2}$, Yejin Heo$^{1}$, Keith H. Baratz$^{3}$, Ashlie A. Bernhisel$^{3}$, Sanjay V. Patel$^{3}$, Amy J. Estes$^{4,5}$, Anthony N. Kuo$^{6}$, and Yutao Liu$^{2,3,5,7}$}

\affiliations{
$^{1}$ Medical College of Georgia, Augusta University, Augusta, GA, USA \\
$^{2}$ Department of Cellular Biology and Anatomy, Augusta University, Augusta, GA, USA \\
$^{3}$ Department of Ophthalmology, Mayo Clinic, Rochester, MN, USA \\
\textit{(Add remaining affiliations as per the image)}
}

\begin{minipage}[t]{0.35\textwidth}
    \infosection{Correspondence}{Yutao Liu, Department of Cellular Biology and Anatomy, Augusta University, 1120 15th St., Augusta, GA 30912, USA. e-mail: yutliu@augusta.edu}
    \infosection{Received}{August 12, 2025}
    \infosection{Accepted}{January 4, 2026}
    \infosection{Published}{February 4, 2026}
    \infosection{Keywords}{keratoconus (KC); PCSK1; genetics; cornea}
    \infosection{Citation}{Beatty C, et al. Transl Vis Sci Technol. 2026;15(2):3.}
\end{minipage}
\hfill
\begin{minipage}[t]{0.60\textwidth}
    \small\rmfamily
    \textbf{Purpose:} Previously, a variant within the \textit{Pcsk1} gene was found to segregate with the keratoconus (KC) phenotype in a four-generation family. We aimed to evaluate a potential relation between the \textit{Pcsk1} gene and corneal phenotype in mouse models.\par\vspace{0.8em}
    \textbf{Methods:} Two strains of \textit{Pcsk1} mice, one with a knockout (KO) and one with an N222D point mutation, were bred...\par\vspace{0.8em}
    \textbf{Results:} No significant differences in corneal CCT, pachymetry, or morphology were observed among any of the mutant mice compared with their control littermates.\par\vspace{0.8em}
    \textbf{Conclusions:} In this setting, neither the N222D point mutation nor the Pcsk1 KO affected the corneal phenotype in two mouse models.
\end{minipage}

\begin{multicols}{2}
\sectionbox{Introduction}


Keratoconus (KC) is a bilateral, asymmetric corneal ectasia characterized by irregular thinning and steepening of the cornea into a conical shape. These progressive changes in the cornea dramatically affect vision, leading to irregular astigmatism, myopia, and decreased visual acuity. Symptoms most often arise in the second or third decade of life...
\end{multicols}

\end{document}